\subsection{Scans}

\begin{frame}[t]{Scan pattern}
\begin{itemize}
  \item A \textmark{scan} pattern computes a sequence of
        partial reductions on a dataset.
    \begin{itemize}
      \item A scan on $x_0$, $x_1$, $x_2$, \ldots 
      \item Results in the sequence:
        \begin{itemize}
          \item $x_0$
          \item $x_0 + x_1$
          \item $x_0 + x_1 + x_2$
          \item \ldots
        \end{itemize}
    \end{itemize}
    \vfill
    \item Two alternatives:
      \begin{itemize}
        \item \cppid{std::exclusive\_scan()}
        \item \cppid{std::inclusive\_scan()}
      \end{itemize}
\end{itemize}
\end{frame}

\begin{frame}[t,fragile]{Computing CDF}
\begin{lstlisting}
auto compute_cdf(const std::vector<int> & histogram)
{
  std::vector<int> cdf(histogram.size());

  std::inclusive_scan(std::execution::par,
    histogram.begin(), histogram.end(),
    cdf.begin(),
    0
  );

  return cdf;
}
\end{lstlisting}
\end{frame}

\begin{frame}[t,fragile]{Combining transform and scan}
\begin{lstlisting}
auto compute_cdf(const std::vector<int> & histogram)
{
  std::vector<int> cdf(histogram.size());

  std::transform_inclusive_scan(std::execution::par,
    histogram.begin(), histogram.end(),
    cdf.begin(),
    0,
    [](auto x, auto y) { return x+y; }
    [](auto x) {
      if (x<0) return 0;
      if (x>255) return 255;
      return x;
    }
  );

  return cdf;
}
\end{lstlisting}
\end{frame}
