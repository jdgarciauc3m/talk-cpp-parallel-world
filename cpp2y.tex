\section{After C++20: Executors}

\begin{frame}[t]{DISCLAIMER}
\begin{quote}
This section contains tentative design that
has currently under discussion.
\end{quote}
\end{frame}

\begin{frame}[t]{Context}
\begin{itemize}
  \item A possible \textmark{future}:
    \begin{itemize}
      \item Composition of networked, asynchronous parallel computations.
      \item Accelerated by diverse hardware
    \end{itemize}
\vfill\pause
  \item But the \textmark{present}:
    \begin{itemize}
      \item Low-level concurrency primitives 
            (\cppid{std::thread}, \cppid{std::atomic}, \ldots).
      \item Components with known problems
            (\cppid{std::async}, \cppid{std::future}, \ldots).
      \item Parallel algorithms neither flexible nor composable.
    \end{itemize}
\vfill\pause
  \item \textgood{Solution} with two components:
    \begin{itemize}
      \item \cppid{executor}s
      \item \cppid{sender}s and \cppid{receiver}s.
    \end{itemize}
\end{itemize}
\end{frame}

\begin{frame}[t,fragile]{Executors}
\begin{itemize}
  \item Executors:
    \begin{itemize}
      \item A work execution interface.
        \begin{itemize}
          \item Any \cppid{executor} type.
        \end{itemize}
\begin{lstlisting}
using namespace std::execution;
std::static_thread_pool p(16);
executor auto ex = p.executor();
execute(ex, []{ do_the_work(); });
\end{lstlisting}
    \end{itemize}
\end{itemize}
\end{frame}

\begin{frame}[t,fragile]{Senders and receivers}
\begin{itemize}
  \item Senders and receivers:
    \begin{itemize}
      \item A representation of work and interrelationships.
        \begin{itemize}
          \item \cppid{sender} types.
          \item \cppid{receiver} types.
        \end{itemize}
\begin{lstlisting}
sender auto begin = schedule(ex);
sender auto next = then(begin, [] { f(); return 42; });
sender auto job = then(next, [](int x) { g(x); return 99; });

receiver auto doit = as_receier([](int x) { store(x); });
submit(job,doit);
\end{lstlisting}
    \end{itemize}
\end{itemize}
\end{frame}

\begin{frame}[t]{What is an executor}
\begin{itemize}
  \item A lightweight handle to an execution context.
    \begin{itemize}
      \item A thread pool.
      \item SIMD units.
      \item GPUs.
      \item Current thread.
      \item \ldots
    \end{itemize}
\end{itemize}
\end{frame}

\begin{frame}[t,fragile]{The simplest executor}
\begin{itemize}
  \item An inline executor executes the work immediately.
\end{itemize}
\begin{lstlisting}
struct inline_executor {
  template<class F>
  void execute(F&& f) const noexcept {
    std::invoke(std::forward<F>(f));
  }

  auto operator<=>(const inline_executor&) const = default;
};
\end{lstlisting}
\end{frame}

\begin{frame}[t,fragile]{Bulk execution}
\begin{itemize}
\item Another example of control structure provided by
      an executor.
  \begin{itemize}
    \item Creates a group of functions calls in a single operation.
  \end{itemize}
\end{itemize}
\begin{lstlisting}
struct simd_executor : inline_executor { 
  template<class F>
  simd_sender bulk_execute(F f, size_t n) const {
    #pragma simd
    for(size_t i = 0; i != n; ++i) {
      std::invoke(f, i);
    }

    return {};
  }
};
\end{lstlisting}
\end{frame}

\begin{frame}[t,fragile]{An executor based for-each}
\begin{lstlisting}
template<class Executor, class F, class Range>
void my_for_each(const Executor& ex, F f, Range rng) {
  // request bulk execution, receive a sender
  sender auto s = execution::bulk_execute(ex, 
    [=](size_t i) {
      f(rng[i]);
    }
  );

  // initiate execution and wait for it to complete
  execution::sync_wait(s);
}
\end{lstlisting}
\end{frame}

\begin{frame}[t,fragile]{A future asynchronous STL?}
\begin{lstlisting}
sender auto s = 
  just(3) |                            // produce '3' immediately
  via(scheduler1) |                    // transition context
  then([](int a){return a+1;}) |       // chain continuation
  then([](int a){return a*2;}) |       // chain another continuation
  via(scheduler2) |                    // transition context
  handle_error([](auto e){
    return just(3);});                 // with default value on errors

int r = sync_wait(s);                  // wait for the result
\end{lstlisting}
\end{frame}
