\subsection{Execution policies}

\begin{frame}[t]{Overview of execution policies}
\vspace{-1em}
\begin{itemize}
  \item \cppid{std::execution::seq}.
    \begin{itemize}
      \item Class \cppid{std::execution::sequenced\_policy}.
      \item Algorithm executes sequentially (single thread).
      \item Might have changes over traditional algorithm.
    \end{itemize}
\pause
  \item \cppid{std::execution::par}.
    \begin{itemize}
      \item Class \cppid{std::execution::parallel\_policy}.
      \item Algorithm executes in multiple threads.
      \item No vectorization!
    \end{itemize}
\pause
  \item \cppid{std::execution::par\_unseq}.
    \begin{itemize}
      \item Class \cppid{std::execution::parallel\_unsequenced\_policy}.
      \item Algorithm executes in multiple threads.
      \item Vectorization allowed!
    \end{itemize}
\pause
  \item \cppid{std::execution::unseq} (\textbf{\textcolor{red}{C++20}}).
    \begin{itemize}
      \item Class \cppid{std::execution::unsequenced\_policy}.
      \item Algorithm executes in single thread.
      \item Vectorization allowed!
    \end{itemize}
\end{itemize}
\end{frame}

\begin{frame}[t,fragile]{Constraints on iterators}
\begin{itemize}
  \item Some algorithms on the STL require ranges expressed
        as \textmark{input iterators}.
\begin{lstlisting}
template< class InputIt, class T >
typename iterator_traits<InputIt>::difference_type
  count(InputIt first, InputIt last, const T &value );
\end{lstlisting}

  \item Execution policy based require iterators to be
        \textmark{forward iterators}
\begin{lstlisting}
template< class ExecutionPolicy, class ForwardIt, class T >
typename iterator_traits<ForwardIt>::difference_type
  count(ExecutionPolicy&& policy, 
        ForwardIt first, ForwardIt last, const T &value );
\end{lstlisting}
\end{itemize}
\end{frame}

\begin{frame}[t,fragile]{Changes in algorithms interface}
\begin{itemize}
  \item Some algorithms have changed their return types.
  \vfill\pause
  \item Without execution policy.
    \begin{itemize}
      \item Returns the comparator object.
    \end{itemize}
\begin{lstlisting}
template <class InputIt, class UnaryFunction>
constexpr UnaryFunction 
  for_each(InputIt first, InputIt last, UnaryFunction f);
\end{lstlisting}
  \vfill\pause
  \item With execution policy.
    \begin{itemize}
      \item Does not return any value.
    \end{itemize}
\begin{lstlisting}[escapeinside={<@}{@>}]
template <class ExecutionPolicy, class ForwardIt, 
          class UnaryFunction2>
<@\textcolor{red}{void}@>
  for_each(ExecutionPolicy&& policy, 
           ForwardIt first, ForwardIt last, UnaryFunction2 f);
\end{lstlisting}
\end{itemize}
\end{frame}

\begin{frame}[t,fragile]{What about exceptions?}
\begin{itemize}
  \item In non execution policy based exceptions can be thrown.
\begin{lstlisting}
std::for_each(v.begin(), v.end(),
  [](auto & x) {
    if (valid(x)} f(x);
    else throw invalid_value{x}; // Throws exception
  });
\end{lstlisting}

\vfill\pause
  \item In excecution policy based exceptions translate into 
        \cppid{std::terminate}.

\begin{lstlisting}
std::for_each(std::execution::seq, v.begin(), v.end(),
  [](auto & x) {
    if (valid(x)} f(x);
    else throw invalid_value{x}; // Invoke std::terminate
  });
\end{lstlisting}
\end{itemize}
\end{frame}

\begin{frame}[t]{When to avoid execution policies}
\begin{itemize}
  \item Using input or output iterators.
  \vfill\pause
  \item Avoid calling \cppid{std::terminate} on exceptions.
  \vfill\pause
  \item Avoid side effects on use of elements.
  \vfill\pause
  \item Make use of return values (e.g. \cppid{std::for\_each()}.
\end{itemize}
\end{frame}
