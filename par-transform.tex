\subsection{Transformations}

\begin{frame}[t]{The map-pattern}
\begin{itemize}
  \item A well known pattern in functional programming.
    \begin{itemize}
      \item Apply an operation to every element in a data set
            to generate a new data set.
    \end{itemize}
\end{itemize}
\end{frame}

\begin{frame}[t,fragile]{Squaring values}
\begin{lstlisting}
std::vector<double> square(const std::vector<double> & v) 
{
  std::vector<double> r(v.size());

  std::transform(std::sequential::par,
    v.begin(), v.end(), r.begin(),
    [](double x) { return x*x; }
  );

  return r;
}
\end{lstlisting}
\end{frame}

\begin{frame}[t,fragile]{Adding vectors}
\begin{lstlisting}
std::vector<double> add(const std::vector<double> & v,
                           const std::vector<double> & w) 
{
  std::vector<double> r(v.size());

  std::transform(std::sequential::par,
    v.begin(), v.end(), w.begin(), r.begin(),
    [](double x, double y) { return x+y; }
  );

  return r;
}
\end{lstlisting}
\end{frame}

\begin{frame}[t,fragile]{Heterogeneous transformations}
\begin{lstlisting}
std::vector<std::complex<double>> create_cplx(
    const std::vector<double> & re,
    const std::vector<double> & im)
{
  auto sz = std::min(re.size(), im.size());
  std::vector<std::complex<double>> res(sz);
  
  std::transform(std::execution::par,
    re.begin(), re.end(), im.begin(),
    res.begin(),
    [](double r, double i) -> complex<double> {
      return {r,i};
    });
  return res;
}
\end{lstlisting}
\end{frame}
